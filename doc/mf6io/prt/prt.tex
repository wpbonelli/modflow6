The PRT Model performs three-dimensional particle tracking in flowing groundwater. ...  

This section describes the data files for a \mf Particle Tracking (PRT) Model.  A PRT Model is added to the simulation by including a PRT entry in the MODELS block of the simulation name file.  There are currently two types of spatial discretization approaches that can be used with the PRT Model: DIS and DISV.  The input instructions for these three packages are not described here in this section on PRT Model input; input instructions for these three packages are described in the section on GWF Model input.

The PRT Model is designed to permit input to be gathered, as it is needed, from many different files.  Likewise, results from the model calculations can be written to a number of output files. ...  Details about the files used by each package are provided in this section on the PRT Model Instructions.

The PRT Model reads a file called the Name File, which specifies most of the files that will be used in a simulation. Several files are always required whereas other files are optional depending on the simulation. The Output Control Package receives instructions from the user to control the amount and frequency of output.  Details about the Name File and the Output Control Package are described in this section.

For the PRT Model, ``flows'' (unless stated otherwise) represent particle mass ``flow'' in mass per time, rather than groundwater flow.  In this implementation, each particle is assigned unit mass, and the numerical value of the flow can be interpreted as particles per time.

\begin{enumerate}

\item The PRT Model simulates transport of ...; however, because \mf allows for multiple models of the same type to be included in a single simulation, ... can be represented by using multiple PRT Models.

\item The PRT Model requires simulated groundwater flows as input. Simulated flows from the GWF Model can be passed in memory to the PRT Model in the same simulation via a GWF-PRT Exchange.  Alternatively, the PRT Model can read binary flow and head files saved by a previously run GWF Model.  The current implemention of the PRT Model does not support particle tracking through the advanced stress packages or the Water Mover Package.

\item Although there is GWF-GWF Exchange, a PRT-PRT Exchange has not yet been developed to connect multiple particle-tracking models, as might be done in a nested grid configuration.  

\end{enumerate}

\subsection{Units of Length and Time}
The GWF Model formulates the groundwater flow equation without using prescribed length and time units. Any consistent units of length and time can be used when specifying the input data for a simulation. This capability gives a certain amount of freedom to the user, but care must be exercised to avoid mixing units.  The program cannot detect the use of inconsistent units.

\subsection{Particle Mass Budget}
A summary of all inflow (sources) and outflow (sinks) of particle mass is called a mass budget.  \mf calculates a mass budget for the overall model as a check on the acceptability of the solution, and to provide a summary of the sources and sinks of mass to the flow system.  The particle mass budget is printed to the PRT Model Listing File for selected time steps.  In the current implementation, each particle is assigned unit mass, and the numerical value of the flow can be interpreted as particles per time.

\subsection{Time Stepping}
In \mf time step lengths are controlled by the user and specified in the Temporal Discretization (TDIS) input file.  When the particle-tracking model and transport model are included in the same simulation, then the length of the time step specified in TDIS is used for both models.  If the PRT Model runs in a separate simulation from the GWF Model, then ....  Instructions for specifying time steps are described in the TDIS section of this user guide; additional information on GWF and PRT configurations are in the Flow Model Interface section.  



\newpage
\subsection{PRT Model Name File}
\input{prt/namefile.tex}

%\newpage
%\subsection{Structured Discretization (DIS) Input File}
%\input{gwf/dis}

%\newpage
%\subsection{Discretization with Vertices (DISV) Input File}
%\input{gwf/disv}

%\newpage
%\subsection{Unstructured Discretization (DISU) Input File}
%\input{gwf/disu}

\newpage
\subsection{Model Input (MIP) Package}
\input{prt/mip}

\newpage
\subsection{Particle Release Point Conditions (PRP) Package}
\input{prt/prp}

\newpage
\subsection{Output Control (OC) Option}
\input{prt/oc}

\newpage
\subsection{Observation (OBS) Utility for a PRT Model}
\input{prt/prt-obs}

\newpage
\subsection{Flow Model Interface (FMI) Package}
\input{prt/fmi}
